\section{Conclusion}

In this study, we have proposed a game theoretic model for the spreading of anti-vaccine sentiments in a network. We have discussed theoretically about the the structure of Nash Equilibrium and which factors matter most in the decision making. We have done experiments on the Small World Network and also on the Facebook Combined Dataset from Stanford Large Network Datasets (SNAP), which show the nature of the Nash Equillibrium and the nature of the anti-vaccine nodes in NE.


Some additional factors that could be included in the model in the future are:

\begin{itemize}
    \item  Some people cannot be vaccinated; they rely solely on herd immunity.
    
    \item The people in the model are not stationary, rather they can move over time and come in contact with other people from whom they can catch the disease even if their neighborhood is free from that disease. It is important to take into account such scenarios in the model.
    
    \item The price-demand relationship of vaccines is an important economic aspect of this problem. If the number of people taking the vaccines increases, the cost associated with taking the vaccines will also increase. Thus, some people will not be able to afford the vaccines.
    
\end{itemize}
\endinput